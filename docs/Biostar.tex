\documentclass[]{ctexbook}
\usepackage{lmodern}
\usepackage{amssymb,amsmath}
\usepackage{ifxetex,ifluatex}
\usepackage{fixltx2e} % provides \textsubscript
\ifnum 0\ifxetex 1\fi\ifluatex 1\fi=0 % if pdftex
  \usepackage[T1]{fontenc}
  \usepackage[utf8]{inputenc}
\else % if luatex or xelatex
  \ifxetex
    \usepackage{xltxtra,xunicode}
  \else
    \usepackage{fontspec}
  \fi
  \defaultfontfeatures{Ligatures=TeX,Scale=MatchLowercase}
\fi
% use upquote if available, for straight quotes in verbatim environments
\IfFileExists{upquote.sty}{\usepackage{upquote}}{}
% use microtype if available
\IfFileExists{microtype.sty}{%
\usepackage{microtype}
\UseMicrotypeSet[protrusion]{basicmath} % disable protrusion for tt fonts
}{}
\usepackage[b5paper,tmargin=2.5cm,bmargin=2.5cm,lmargin=3.5cm,rmargin=2.5cm]{geometry}
\usepackage[unicode=true]{hyperref}
\PassOptionsToPackage{usenames,dvipsnames}{color} % color is loaded by hyperref
\hypersetup{
            pdftitle={Biostar},
            pdfauthor={小苏},
            colorlinks=true,
            linkcolor=Maroon,
            citecolor=Blue,
            urlcolor=Blue,
            breaklinks=true}
\urlstyle{same}  % don't use monospace font for urls
\usepackage{natbib}
\bibliographystyle{apalike}
\usepackage{color}
\usepackage{fancyvrb}
\newcommand{\VerbBar}{|}
\newcommand{\VERB}{\Verb[commandchars=\\\{\}]}
\DefineVerbatimEnvironment{Highlighting}{Verbatim}{commandchars=\\\{\}}
% Add ',fontsize=\small' for more characters per line
\usepackage{framed}
\definecolor{shadecolor}{RGB}{248,248,248}
\newenvironment{Shaded}{\begin{snugshade}}{\end{snugshade}}
\newcommand{\AlertTok}[1]{\textcolor[rgb]{0.94,0.16,0.16}{#1}}
\newcommand{\AnnotationTok}[1]{\textcolor[rgb]{0.56,0.35,0.01}{\textbf{\textit{#1}}}}
\newcommand{\AttributeTok}[1]{\textcolor[rgb]{0.13,0.29,0.53}{#1}}
\newcommand{\BaseNTok}[1]{\textcolor[rgb]{0.00,0.00,0.81}{#1}}
\newcommand{\BuiltInTok}[1]{#1}
\newcommand{\CharTok}[1]{\textcolor[rgb]{0.31,0.60,0.02}{#1}}
\newcommand{\CommentTok}[1]{\textcolor[rgb]{0.56,0.35,0.01}{\textit{#1}}}
\newcommand{\CommentVarTok}[1]{\textcolor[rgb]{0.56,0.35,0.01}{\textbf{\textit{#1}}}}
\newcommand{\ConstantTok}[1]{\textcolor[rgb]{0.56,0.35,0.01}{#1}}
\newcommand{\ControlFlowTok}[1]{\textcolor[rgb]{0.13,0.29,0.53}{\textbf{#1}}}
\newcommand{\DataTypeTok}[1]{\textcolor[rgb]{0.13,0.29,0.53}{#1}}
\newcommand{\DecValTok}[1]{\textcolor[rgb]{0.00,0.00,0.81}{#1}}
\newcommand{\DocumentationTok}[1]{\textcolor[rgb]{0.56,0.35,0.01}{\textbf{\textit{#1}}}}
\newcommand{\ErrorTok}[1]{\textcolor[rgb]{0.64,0.00,0.00}{\textbf{#1}}}
\newcommand{\ExtensionTok}[1]{#1}
\newcommand{\FloatTok}[1]{\textcolor[rgb]{0.00,0.00,0.81}{#1}}
\newcommand{\FunctionTok}[1]{\textcolor[rgb]{0.13,0.29,0.53}{\textbf{#1}}}
\newcommand{\ImportTok}[1]{#1}
\newcommand{\InformationTok}[1]{\textcolor[rgb]{0.56,0.35,0.01}{\textbf{\textit{#1}}}}
\newcommand{\KeywordTok}[1]{\textcolor[rgb]{0.13,0.29,0.53}{\textbf{#1}}}
\newcommand{\NormalTok}[1]{#1}
\newcommand{\OperatorTok}[1]{\textcolor[rgb]{0.81,0.36,0.00}{\textbf{#1}}}
\newcommand{\OtherTok}[1]{\textcolor[rgb]{0.56,0.35,0.01}{#1}}
\newcommand{\PreprocessorTok}[1]{\textcolor[rgb]{0.56,0.35,0.01}{\textit{#1}}}
\newcommand{\RegionMarkerTok}[1]{#1}
\newcommand{\SpecialCharTok}[1]{\textcolor[rgb]{0.81,0.36,0.00}{\textbf{#1}}}
\newcommand{\SpecialStringTok}[1]{\textcolor[rgb]{0.31,0.60,0.02}{#1}}
\newcommand{\StringTok}[1]{\textcolor[rgb]{0.31,0.60,0.02}{#1}}
\newcommand{\VariableTok}[1]{\textcolor[rgb]{0.00,0.00,0.00}{#1}}
\newcommand{\VerbatimStringTok}[1]{\textcolor[rgb]{0.31,0.60,0.02}{#1}}
\newcommand{\WarningTok}[1]{\textcolor[rgb]{0.56,0.35,0.01}{\textbf{\textit{#1}}}}
\usepackage{longtable,booktabs}
% Fix footnotes in tables (requires footnote package)
\IfFileExists{footnote.sty}{\usepackage{footnote}\makesavenoteenv{long table}}{}
\IfFileExists{parskip.sty}{%
\usepackage{parskip}
}{% else
\setlength{\parindent}{0pt}
\setlength{\parskip}{6pt plus 2pt minus 1pt}
}
\setlength{\emergencystretch}{3em}  % prevent overfull lines
\providecommand{\tightlist}{%
  \setlength{\itemsep}{0pt}\setlength{\parskip}{0pt}}
\setcounter{secnumdepth}{5}
% Redefines (sub)paragraphs to behave more like sections
\ifx\paragraph\undefined\else
\let\oldparagraph\paragraph
\renewcommand{\paragraph}[1]{\oldparagraph{#1}\mbox{}}
\fi
\ifx\subparagraph\undefined\else
\let\oldsubparagraph\subparagraph
\renewcommand{\subparagraph}[1]{\oldsubparagraph{#1}\mbox{}}
\fi

% set default figure placement to htbp
\makeatletter
\def\fps@figure{htbp}
\makeatother

\usepackage{booktabs}
\usepackage{longtable}

\usepackage{framed,color}
\definecolor{shadecolor}{RGB}{248,248,248}

\renewcommand{\textfraction}{0.05}
\renewcommand{\topfraction}{0.8}
\renewcommand{\bottomfraction}{0.8}
\renewcommand{\floatpagefraction}{0.75}

\let\oldhref\href
\renewcommand{\href}[2]{#2\footnote{\url{#1}}}

\makeatletter
\newenvironment{kframe}{%
\medskip{}
\setlength{\fboxsep}{.8em}
 \def\at@end@of@kframe{}%
 \ifinner\ifhmode%
  \def\at@end@of@kframe{\end{minipage}}%
  \begin{minipage}{\columnwidth}%
 \fi\fi%
 \def\FrameCommand##1{\hskip\@totalleftmargin \hskip-\fboxsep
 \colorbox{shadecolor}{##1}\hskip-\fboxsep
     % There is no \\@totalrightmargin, so:
     \hskip-\linewidth \hskip-\@totalleftmargin \hskip\columnwidth}%
 \MakeFramed {\advance\hsize-\width
   \@totalleftmargin\z@ \linewidth\hsize
   \@setminipage}}%
 {\par\unskip\endMakeFramed%
 \at@end@of@kframe}
\makeatother

\makeatletter
\@ifundefined{Shaded}{
}{\renewenvironment{Shaded}{\begin{kframe}}{\end{kframe}}}
\@ifpackageloaded{fancyvrb}{%
  % https://github.com/CTeX-org/ctex-kit/issues/331
  \RecustomVerbatimEnvironment{Highlighting}{Verbatim}{commandchars=\\\{\},formatcom=\xeCJKVerbAddon}%
}{}
\makeatother

\usepackage{makeidx}
\makeindex

\urlstyle{tt}

\usepackage{amsthm}
\makeatletter
\def\thm@space@setup{%
  \thm@preskip=8pt plus 2pt minus 4pt
  \thm@postskip=\thm@preskip
}
\makeatother

\newenvironment{cols}[1][]{}{}

\newenvironment{col}[1]{\begin{minipage}{#1}\ignorespaces}{%
	\end{minipage}
	\ifhmode\unskip\fi
	\aftergroup\useignorespacesandallpars}

\def\useignorespacesandallpars#1\ignorespaces\fi{%
	#1\fi\ignorespacesandallpars}

\makeatletter
\def\ignorespacesandallpars{%
	\@ifnextchar\par
	{\expandafter\ignorespacesandallpars\@gobble}%
	{}%
}
\makeatother

\frontmatter
\usepackage{booktabs}
\usepackage{longtable}
\usepackage{array}
\usepackage{multirow}
\usepackage{wrapfig}
\usepackage{float}
\usepackage{colortbl}
\usepackage{pdflscape}
\usepackage{tabu}
\usepackage{threeparttable}
\usepackage{threeparttablex}
\usepackage[normalem]{ulem}
\usepackage{makecell}
\usepackage{xcolor}

\title{Biostar}
\author{小苏}
\date{2025-03-31}

\begin{document}
\maketitle

\thispagestyle{empty}

\begin{center}
献给……
\end{center}

\setlength{\abovedisplayskip}{-5pt}
\setlength{\abovedisplayshortskip}{-5pt}

{
\setcounter{tocdepth}{1}
\tableofcontents
}
\listoftables
\listoffigures
\chapter*{前言}\label{ux524dux8a00}


\emph{我不是代码的创作者,我只是代码的搬运工!}

\part{数据下载}\label{part-ux6570ux636eux4e0bux8f7d}

\chapter{aria2}\label{aria2}

aria2 是一个轻量级的多协议和多源命令行下载工具。它支持 HTTP/HTTPS、FTP、BitTorrent 和 Metalink。aria2 可以通过最大化网络带宽利用率来加快下载速度。它支持 HTTP/HTTPS 代理,SOCKS 代理,HTTP 代理隧道,NAT 穿透,IPv6 和 IP 版本选择。aria2 可以通过 JSON-RPC 和 XML-RPC 接口进行控制。

官方教程:\url{https://github.com/aria2/aria2}

使用示范

\begin{Shaded}
\begin{Highlighting}[]
\VariableTok{URL}\OperatorTok{=}\NormalTok{https://ftp.ensembl.org/pub/release{-}113/fasta/homo\_sapiens/dna/Homo\_sapiens.GRCh38.dna.toplevel.fa.gz}

\FunctionTok{make} \AttributeTok{{-}f}\NormalTok{ src/run/aria.mk URL=}\VariableTok{$\{URL\}}\NormalTok{ run}
\end{Highlighting}
\end{Shaded}

\href{src/run/aria.mk}{aria.mk}

\chapter{curl}\label{curl}

curl 是一个命令行工具和库,用于传输数据,支持多种协议,包括 HTTP、HTTPS、FTP、FTPS、SFTP、IMAP、SMTP、POP3、LDAP、RTMP 和 RTSP。curl 还支持 SSL 证书、HTTP POST、HTTP PUT、FTP 上传、HTTP 基本身份验证、代理、cookie、用户代理、压缩、断点续传、文件传输限速、重定向等功能。

使用示范:

\begin{Shaded}
\begin{Highlighting}[]
\VariableTok{URL}\OperatorTok{=}\NormalTok{https://ftp.ensembl.org/pub/release{-}113/fasta/homo\_sapiens/dna/Homo\_sapiens.GRCh38.dna.toplevel.fa.gz}

\FunctionTok{make} \AttributeTok{{-}f}\NormalTok{ src/run/curl.mk URL=}\VariableTok{$\{URL\}}\NormalTok{ run}
\end{Highlighting}
\end{Shaded}

\href{src/run/curl.mk}{aria.mk}

\chapter{SRA}\label{sra}

SRA(Sequence Read Archive) 是一个由美国国家生物技术信息中心(NCBI)维护的高通量测序数据公共仓库。它是国际核苷酸序列数据库合作(INSDC)的组成部分,与欧洲生物信息研究所(EBI)和日本DNA数据库(DDBJ)共享数据。从数据库中下载原始测序数据是必备技能。

使用示范:
1. 使用 fastq-dump 工具。

\begin{Shaded}
\begin{Highlighting}[]
\VariableTok{SRR}\OperatorTok{=}\NormalTok{SRR12351448}

\FunctionTok{make} \AttributeTok{{-}f}\NormalTok{ src/run/sra.mk SRR=}\VariableTok{$\{SRR\}}\NormalTok{ N=ALL run}
\end{Highlighting}
\end{Shaded}

\begin{Shaded}
\begin{Highlighting}[]
\VariableTok{design}\OperatorTok{=}\NormalTok{design.csv}

\FunctionTok{cat} \VariableTok{$\{design\}} \KeywordTok{|} \ExtensionTok{parallel} \AttributeTok{{-}v} \AttributeTok{{-}{-}eta} \AttributeTok{{-}{-}lb} \AttributeTok{{-}{-}header}\NormalTok{ : }\AttributeTok{{-}{-}colsep}\NormalTok{ , }\DataTypeTok{\textbackslash{}}
\NormalTok{                make }\AttributeTok{{-}f}\NormalTok{ src/run/sra.mk }\DataTypeTok{\textbackslash{}}
\NormalTok{                SRR=\{Run\} }\DataTypeTok{\textbackslash{}}
\NormalTok{                N=ALL }\DataTypeTok{\textbackslash{}}
\NormalTok{                run}
\end{Highlighting}
\end{Shaded}

\begin{enumerate}
\def\labelenumi{\arabic{enumi}.}
\setcounter{enumi}{1}
\tightlist
\item
  使用 aria2 工具。
\end{enumerate}

\begin{Shaded}
\begin{Highlighting}[]
\VariableTok{SRR}\OperatorTok{=}\NormalTok{SRR12351448}

\FunctionTok{make} \AttributeTok{{-}f}\NormalTok{ src/run/sra.mk SRR=}\VariableTok{$\{SRR\}}\NormalTok{ N=ALL aria}
\end{Highlighting}
\end{Shaded}

\begin{Shaded}
\begin{Highlighting}[]
\VariableTok{design}\OperatorTok{=}\NormalTok{design.csv}

\FunctionTok{cat} \VariableTok{$\{design\}} \KeywordTok{|} \ExtensionTok{parallel} \AttributeTok{{-}v} \AttributeTok{{-}{-}eta} \AttributeTok{{-}{-}lb} \AttributeTok{{-}{-}header}\NormalTok{ : }\AttributeTok{{-}{-}colsep}\NormalTok{ , }\DataTypeTok{\textbackslash{}}
\NormalTok{                make }\AttributeTok{{-}f}\NormalTok{ src/run/sra.mk }\DataTypeTok{\textbackslash{}}
\NormalTok{                SRR=\{Run\} }\DataTypeTok{\textbackslash{}}
\NormalTok{                N=ALL }\DataTypeTok{\textbackslash{}}
\NormalTok{                aria}
\end{Highlighting}
\end{Shaded}

\href{src/run/sra.mk}{sra.mk}

\part{fastq 质量控制}\label{part-fastq-ux8d28ux91cfux63a7ux5236}

\chapter{fastp}\label{fastp}

官方教程:\url{https://github.com/OpenGene/fastp}

使用示范:

单端数据

\begin{Shaded}
\begin{Highlighting}[]
\VariableTok{SRR}\OperatorTok{=}\NormalTok{SRR12351448}
\VariableTok{DATA}\OperatorTok{=}\NormalTok{data}
\VariableTok{READS}\OperatorTok{=}\VariableTok{$\{DATA\}}\NormalTok{/reads}

\FunctionTok{make} \AttributeTok{{-}f}\NormalTok{ src/run/fastp.mk P1=}\VariableTok{$\{READS\}}\NormalTok{/}\VariableTok{$\{SRR\}}\NormalTok{\_1.fastq run}
\end{Highlighting}
\end{Shaded}

\begin{Shaded}
\begin{Highlighting}[]
\VariableTok{DATA}\OperatorTok{=}\NormalTok{data}
\VariableTok{READS}\OperatorTok{=}\VariableTok{$\{DATA\}}\NormalTok{/reads}
\VariableTok{design}\OperatorTok{=}\VariableTok{$\{DATA\}}\NormalTok{/design.csv}

\FunctionTok{cat} \VariableTok{$\{design\}} \KeywordTok{|} \ExtensionTok{parallel} \AttributeTok{{-}{-}header}\NormalTok{ : }\AttributeTok{{-}{-}colsep}\NormalTok{ , }\DataTypeTok{\textbackslash{}}
\NormalTok{        make }\AttributeTok{{-}f}\NormalTok{ src/run/fastp.mk }\DataTypeTok{\textbackslash{}}
\NormalTok{        P1=}\VariableTok{$\{READS\}}\NormalTok{/\{Run\}\_1.fastq }\DataTypeTok{\textbackslash{}}
\NormalTok{        run}
\end{Highlighting}
\end{Shaded}

双端数据

\begin{Shaded}
\begin{Highlighting}[]
\VariableTok{SRR}\OperatorTok{=}\NormalTok{SRR12351448}
\VariableTok{DATA}\OperatorTok{=}\NormalTok{data}
\VariableTok{READS}\OperatorTok{=}\VariableTok{$\{DATA\}}\NormalTok{/reads}

\FunctionTok{make} \AttributeTok{{-}f}\NormalTok{ src/run/fastp.mk P1=}\VariableTok{$\{READS\}}\NormalTok{/}\VariableTok{$\{SRR\}}\NormalTok{\_1.fastq P2=}\VariableTok{$\{READS\}}\NormalTok{/}\VariableTok{$\{SRR\}}\NormalTok{\_2.fastq run}
\end{Highlighting}
\end{Shaded}

\begin{Shaded}
\begin{Highlighting}[]
\VariableTok{DATA}\OperatorTok{=}\NormalTok{data}
\VariableTok{READS}\OperatorTok{=}\VariableTok{$\{DATA\}}\NormalTok{/reads}
\VariableTok{design}\OperatorTok{=}\VariableTok{$\{DATA\}}\NormalTok{/design.csv}

\FunctionTok{cat} \VariableTok{$\{design\}} \KeywordTok{|} \ExtensionTok{parallel} \AttributeTok{{-}{-}header}\NormalTok{ : }\AttributeTok{{-}{-}colsep}\NormalTok{ , }\DataTypeTok{\textbackslash{}}
\NormalTok{        make }\AttributeTok{{-}f}\NormalTok{ src/run/fastp.mk }\DataTypeTok{\textbackslash{}}
\NormalTok{        P1=}\VariableTok{$\{READS\}}\NormalTok{/\{Run\}\_1.fastq }\DataTypeTok{\textbackslash{}}
\NormalTok{        P2=}\VariableTok{$\{READS\}}\NormalTok{/\{Run\}\_2.fastq }\DataTypeTok{\textbackslash{}}
\NormalTok{        run}
\end{Highlighting}
\end{Shaded}

\href{src/run/fastp.mk}{sra.mk}

\part{比对}\label{part-ux6bd4ux5bf9}

\chapter{bwa}\label{bwa}

官方教程:\url{https://github.com/lh3/bwa}

\section{构建索引}\label{ux6784ux5efaux7d22ux5f15}

\begin{Shaded}
\begin{Highlighting}[]
\VariableTok{REF}\OperatorTok{=}\NormalTok{\textasciitilde{}/database/Human/Homo\_sapiens.GRCh38.dna.toplevel.fa}

\FunctionTok{make} \AttributeTok{{-}f}\NormalTok{ \textasciitilde{}/src/run/bwa.mk REF=}\VariableTok{$\{REF\}}\NormalTok{ index}
\end{Highlighting}
\end{Shaded}

\section{比对}\label{ux6bd4ux5bf9}

单端数据

\begin{Shaded}
\begin{Highlighting}[]
\VariableTok{DATA}\OperatorTok{=}\NormalTok{data}
\VariableTok{REF}\OperatorTok{=}\NormalTok{genome/Human/Homo\_sapiens.GRCh38.dna.primary\_assembly.fa}
\VariableTok{BWA}\OperatorTok{=}\NormalTok{results/bwa}
\VariableTok{CLEANDATA}\OperatorTok{=}\VariableTok{$\{DATA\}}\NormalTok{/reads}
\VariableTok{Run}\OperatorTok{=}\NormalTok{SRR1343245}

\FunctionTok{make} \AttributeTok{{-}f}\NormalTok{ src/run/bwa.mk }\DataTypeTok{\textbackslash{}}
\NormalTok{        REF=}\VariableTok{$\{REF\}} \DataTypeTok{\textbackslash{}}
\NormalTok{        R1=}\VariableTok{$\{CLEANDATA\}}\NormalTok{/}\VariableTok{$\{Run\}}\NormalTok{\_1.trimmed.fastq }\DataTypeTok{\textbackslash{}}
\NormalTok{        BAM=}\VariableTok{$\{BWA\}}\NormalTok{/\{sample\}.bam }\DataTypeTok{\textbackslash{}}
\NormalTok{        run}
\end{Highlighting}
\end{Shaded}

\begin{Shaded}
\begin{Highlighting}[]
\VariableTok{DATA}\OperatorTok{=}\NormalTok{data}
\VariableTok{design}\OperatorTok{=}\VariableTok{$\{DATA\}}\NormalTok{/design.csv}
\VariableTok{REF}\OperatorTok{=}\NormalTok{genome/Human/Homo\_sapiens.GRCh38.dna.primary\_assembly.fa}
\VariableTok{CLEANDATA}\OperatorTok{=}\VariableTok{$\{DATA\}}\NormalTok{/reads}
\VariableTok{BWA}\OperatorTok{=}\NormalTok{results/bwa}
                
\FunctionTok{cat} \VariableTok{$\{design\}} \KeywordTok{|} \ExtensionTok{parallel} \AttributeTok{{-}{-}header}\NormalTok{ : }\AttributeTok{{-}{-}colsep}\NormalTok{ , }\DataTypeTok{\textbackslash{}}
\NormalTok{        make }\AttributeTok{{-}f}\NormalTok{ src/run/bwa.mk }\DataTypeTok{\textbackslash{}}
\NormalTok{        REF=}\VariableTok{$\{REF\}} \DataTypeTok{\textbackslash{}}
\NormalTok{        R1=}\VariableTok{$\{CLEANDATA\}}\NormalTok{/\{Run\}\_1.trimmed.fastq }\DataTypeTok{\textbackslash{}}
\NormalTok{        BAM=}\VariableTok{$\{BWA\}}\NormalTok{/\{sample\}.bam }\DataTypeTok{\textbackslash{}}
\NormalTok{        run}
\end{Highlighting}
\end{Shaded}

双端数据

\begin{Shaded}
\begin{Highlighting}[]
\VariableTok{DATA}\OperatorTok{=}\NormalTok{data}
\VariableTok{REF}\OperatorTok{=}\NormalTok{genome/Human/Homo\_sapiens.GRCh38.dna.primary\_assembly.fa}
\VariableTok{BWA}\OperatorTok{=}\NormalTok{results/bwa}
\VariableTok{CLEANDATA}\OperatorTok{=}\VariableTok{$\{DATA\}}\NormalTok{/reads}
\VariableTok{Run}\OperatorTok{=}\NormalTok{SRR1343245}

\FunctionTok{make} \AttributeTok{{-}f}\NormalTok{ src/run/bwa.mk }\DataTypeTok{\textbackslash{}}
\NormalTok{        REF=}\VariableTok{$\{REF\}} \DataTypeTok{\textbackslash{}}
\NormalTok{        R1=}\VariableTok{$\{CLEANDATA\}}\NormalTok{/}\VariableTok{$\{Run\}}\NormalTok{\_1.trimmed.fastq }\DataTypeTok{\textbackslash{}}
\NormalTok{        R2=}\VariableTok{$\{CLEANDATA\}}\NormalTok{/}\VariableTok{$\{Run\}}\NormalTok{\_2.trimmed.fastq}
        \VariableTok{BAM}\OperatorTok{=}\VariableTok{$\{BWA\}}\NormalTok{/\{sample\}.bam }\DataTypeTok{\textbackslash{}}
        \ExtensionTok{run}
\end{Highlighting}
\end{Shaded}

\begin{Shaded}
\begin{Highlighting}[]
\VariableTok{DATA}\OperatorTok{=}\NormalTok{data}
\VariableTok{design}\OperatorTok{=}\VariableTok{$\{DATA\}}\NormalTok{/design.csv}
\VariableTok{REF}\OperatorTok{=}\NormalTok{genome/Human/Homo\_sapiens.GRCh38.dna.primary\_assembly.fa}
\VariableTok{CLEANDATA}\OperatorTok{=}\VariableTok{$\{DATA\}}\NormalTok{/reads}
\VariableTok{BWA}\OperatorTok{=}\NormalTok{results/bwa}
                
\FunctionTok{cat} \VariableTok{$\{design\}} \KeywordTok{|} \ExtensionTok{parallel} \AttributeTok{{-}{-}header}\NormalTok{ : }\AttributeTok{{-}{-}colsep}\NormalTok{ , }\DataTypeTok{\textbackslash{}}
\NormalTok{        make }\AttributeTok{{-}f}\NormalTok{ src/run/bwa.mk }\DataTypeTok{\textbackslash{}}
\NormalTok{        REF=}\VariableTok{$\{REF\}} \DataTypeTok{\textbackslash{}}
\NormalTok{        R1=}\VariableTok{$\{CLEANDATA\}}\NormalTok{/\{Run\}\_1.trimmed.fastq }\DataTypeTok{\textbackslash{}}
\NormalTok{        R2=}\VariableTok{$\{CLEANDATA\}}\NormalTok{/\{Run\}\_2.trimmed.fastq }\DataTypeTok{\textbackslash{}}
\NormalTok{        BAM=}\VariableTok{$\{BWA\}}\NormalTok{/\{sample\}.bam }\DataTypeTok{\textbackslash{}}
\NormalTok{        run}
\end{Highlighting}
\end{Shaded}

\href{src/run/bwa.mk}{sra.mk}

\chapter{hisat2}\label{hisat2}

官方教程:\url{https://daehwankimlab.github.io/hisat2/}

\section{构建索引}\label{ux6784ux5efaux7d22ux5f15-1}

\begin{Shaded}
\begin{Highlighting}[]
\VariableTok{REF}\OperatorTok{=}\NormalTok{\textasciitilde{}/database/Human/Homo\_sapiens.GRCh38.dna.toplevel.fa}

\FunctionTok{make} \AttributeTok{{-}f}\NormalTok{ \textasciitilde{}/src/run/hisat2.mk REF=}\VariableTok{$\{REF\}}\NormalTok{ index}
\end{Highlighting}
\end{Shaded}

\section{比对}\label{ux6bd4ux5bf9-1}

单端数据

\begin{Shaded}
\begin{Highlighting}[]
\VariableTok{DATA}\OperatorTok{=}\NormalTok{data}
\VariableTok{REF}\OperatorTok{=}\NormalTok{genome/Human/Homo\_sapiens.GRCh38.dna.primary\_assembly.fa}
\VariableTok{HISAT2}\OperatorTok{=}\NormalTok{results/hisat2}
\VariableTok{CLEANDATA}\OperatorTok{=}\VariableTok{$\{DATA\}}\NormalTok{/reads}
\VariableTok{Run}\OperatorTok{=}\NormalTok{SRR1343245}

\FunctionTok{make} \AttributeTok{{-}f}\NormalTok{ src/run/hisat2.mk }\DataTypeTok{\textbackslash{}}
\NormalTok{        REF=}\VariableTok{$\{REF\}} \DataTypeTok{\textbackslash{}}
\NormalTok{        R1=}\VariableTok{$\{CLEANDATA\}}\NormalTok{/}\VariableTok{$\{Run\}}\NormalTok{\_1.trimmed.fastq }\DataTypeTok{\textbackslash{}}
\NormalTok{        BAM=}\VariableTok{$\{HISAT2\}}\NormalTok{/\{sample\}.bam }\DataTypeTok{\textbackslash{}}
\NormalTok{        run}
\end{Highlighting}
\end{Shaded}

\begin{Shaded}
\begin{Highlighting}[]
\VariableTok{DATA}\OperatorTok{=}\NormalTok{data}
\VariableTok{design}\OperatorTok{=}\VariableTok{$\{DATA\}}\NormalTok{/design.csv}
\VariableTok{REF}\OperatorTok{=}\NormalTok{genome/Human/Homo\_sapiens.GRCh38.dna.primary\_assembly.fa}
\VariableTok{CLEANDATA}\OperatorTok{=}\VariableTok{$\{DATA\}}\NormalTok{/reads}
\VariableTok{HISAT2}\OperatorTok{=}\NormalTok{results/hisat2}
                
\FunctionTok{cat} \VariableTok{$\{design\}} \KeywordTok{|} \ExtensionTok{parallel} \AttributeTok{{-}{-}header}\NormalTok{ : }\AttributeTok{{-}{-}colsep}\NormalTok{ , }\DataTypeTok{\textbackslash{}}
\NormalTok{        make }\AttributeTok{{-}f}\NormalTok{ src/run/hisat2.mk }\DataTypeTok{\textbackslash{}}
\NormalTok{        REF=}\VariableTok{$\{REF\}} \DataTypeTok{\textbackslash{}}
\NormalTok{        R1=}\VariableTok{$\{CLEANDATA\}}\NormalTok{/\{Run\}\_1.trimmed.fastq }\DataTypeTok{\textbackslash{}}
\NormalTok{        BAM=}\VariableTok{$\{HISAT2\}}\NormalTok{/\{sample\}.bam }\DataTypeTok{\textbackslash{}}
\NormalTok{        run}
\end{Highlighting}
\end{Shaded}

双端数据

\begin{Shaded}
\begin{Highlighting}[]
\VariableTok{DATA}\OperatorTok{=}\NormalTok{data}
\VariableTok{REF}\OperatorTok{=}\NormalTok{genome/Human/Homo\_sapiens.GRCh38.dna.primary\_assembly.fa}
\VariableTok{HISAT2}\OperatorTok{=}\NormalTok{results/hisat2}
\VariableTok{CLEANDATA}\OperatorTok{=}\VariableTok{$\{DATA\}}\NormalTok{/reads}
\VariableTok{Run}\OperatorTok{=}\NormalTok{SRR1343245}

\FunctionTok{make} \AttributeTok{{-}f}\NormalTok{ src/run/hisat2.mk }\DataTypeTok{\textbackslash{}}
\NormalTok{        REF=}\VariableTok{$\{REF\}} \DataTypeTok{\textbackslash{}}
\NormalTok{        R1=}\VariableTok{$\{CLEANDATA\}}\NormalTok{/}\VariableTok{$\{Run\}}\NormalTok{\_1.trimmed.fastq }\DataTypeTok{\textbackslash{}}
\NormalTok{        R2=}\VariableTok{$\{CLEANDATA\}}\NormalTok{/}\VariableTok{$\{Run\}}\NormalTok{\_2.trimmed.fastq}
        \VariableTok{BAM}\OperatorTok{=}\VariableTok{$\{HISAT2\}}\NormalTok{/\{sample\}.bam }\DataTypeTok{\textbackslash{}}
        \ExtensionTok{run}
\end{Highlighting}
\end{Shaded}

\begin{Shaded}
\begin{Highlighting}[]
\VariableTok{DATA}\OperatorTok{=}\NormalTok{data}
\VariableTok{design}\OperatorTok{=}\VariableTok{$\{DATA\}}\NormalTok{/design.csv}
\VariableTok{REF}\OperatorTok{=}\NormalTok{genome/Human/Homo\_sapiens.GRCh38.dna.primary\_assembly.fa}
\VariableTok{CLEANDATA}\OperatorTok{=}\VariableTok{$\{DATA\}}\NormalTok{/reads}
\VariableTok{HISAT2}\OperatorTok{=}\NormalTok{results/hisat2}
                
\FunctionTok{cat} \VariableTok{$\{design\}} \KeywordTok{|} \ExtensionTok{parallel} \AttributeTok{{-}{-}header}\NormalTok{ : }\AttributeTok{{-}{-}colsep}\NormalTok{ , }\DataTypeTok{\textbackslash{}}
\NormalTok{        make }\AttributeTok{{-}f}\NormalTok{ src/run/hisat2.mk }\DataTypeTok{\textbackslash{}}
\NormalTok{        REF=}\VariableTok{$\{REF\}} \DataTypeTok{\textbackslash{}}
\NormalTok{        R1=}\VariableTok{$\{CLEANDATA\}}\NormalTok{/\{Run\}\_1.trimmed.fastq }\DataTypeTok{\textbackslash{}}
\NormalTok{        R2=}\VariableTok{$\{CLEANDATA\}}\NormalTok{/\{Run\}\_2.trimmed.fastq }\DataTypeTok{\textbackslash{}}
\NormalTok{        BAM=}\VariableTok{$\{HISAT2\}}\NormalTok{/\{sample\}.bam }\DataTypeTok{\textbackslash{}}
\NormalTok{        run}
\end{Highlighting}
\end{Shaded}

\href{src/run/hisat2.mk}{sra.mk}

\part{模块化脚本}\label{part-ux6a21ux5757ux5316ux811aux672c}

\chapter{makefile}\label{makefile}

\section{简介}\label{ux7b80ux4ecb}

官方教程:

\url{https://www.gnu.org/software/make/manual/make.html}

在生信上游分析中,我们不可能每次都手动输入我们的命令,通常我们会将我们的命令写入到一个脚本文件中,然后通过运行脚本文件来执行我们的命令。但是,当我们工作有些变化时,我们就需要做出一些改变,包括增加或删除,这样就会增加我们的工作量和代码复用性。因此,我们可以使用 makefile 来模块化我们的分析流程。

Makefile 是一种用于自动化构建过程的文件,通常用于 Unix 和类 Unix 系统上。它由一个名为 \texttt{make} 的工具读取和执行。Makefile 定义了一系列的规则和依赖关系,用于指导如何编译和链接程序。

在生信分析中使用 makefile 可以帮助我们自动化分析流程,我们不需要特别高级的学习。
下面是一个简单的 makefile 示例:

首先我们创建一个文件,命名为\texttt{Makefile},然后在文件中写入一些内容。

\begin{Shaded}
\begin{Highlighting}[]
\DataTypeTok{.RECIPEPREFIX} \CharTok{=}\StringTok{ \textgreater{}}

\CommentTok{\# 命令 1 }
\DecValTok{foo:}
\NormalTok{\textgreater{} echo Hello world!}

\CommentTok{\# 命令 2}
\DecValTok{bar:}
\NormalTok{\textgreater{} echo Hello world!}
\NormalTok{\textgreater{} echo Hello Everyone!}
\end{Highlighting}
\end{Shaded}

在这个 makefile 中,我们定义了两个目标 \texttt{foo} 和 \texttt{bar},分别对应两个命令。我们可以通过 \texttt{make\ foo} 和 \texttt{make\ bar} 来执行这两个命令。

\begin{Shaded}
\begin{Highlighting}[]
\ExtensionTok{$}\NormalTok{ make }\AttributeTok{{-}f}\NormalTok{ Makefile foo}
\BuiltInTok{echo}\NormalTok{ Hello world!}
\ExtensionTok{Hello}\NormalTok{ world!}
\end{Highlighting}
\end{Shaded}

\begin{Shaded}
\begin{Highlighting}[]
\ExtensionTok{$}\NormalTok{ make }\AttributeTok{{-}f}\NormalTok{ Makefile bar}
\BuiltInTok{echo}\NormalTok{ Hello world!}
\ExtensionTok{Hello}\NormalTok{ world!}
\BuiltInTok{echo}\NormalTok{ Hello Everyone!}
\ExtensionTok{Hello}\NormalTok{ Everyone!}
\end{Highlighting}
\end{Shaded}

当你的文件名是 \texttt{Makefile} 时,你可以直接使用 \texttt{make} 命令,而不需要 \texttt{-f} 参数。

通常 makefile 会将命令和结果同时输出到终端,如果我们只想输出结果,\href{mailto:可以将命令前加一个@符号}{\nolinkurl{可以将命令前加一个@符号}}。比如这样:

\begin{Shaded}
\begin{Highlighting}[]
\DataTypeTok{.RECIPEPREFIX} \CharTok{=}\StringTok{ \textgreater{}}

\CommentTok{\# 命令 1 }
\DecValTok{foo:}
\NormalTok{\textgreater{} @echo Hello world!}
\end{Highlighting}
\end{Shaded}

和 bash 脚本一样,makefile 也可以使用变量,比如:

\begin{Shaded}
\begin{Highlighting}[]
\DataTypeTok{.RECIPEPREFIX} \CharTok{=}\StringTok{ \textgreater{}}

\DataTypeTok{NAME} \CharTok{=}\StringTok{ xiaoming}

\DecValTok{hello:}
\NormalTok{\textgreater{} @echo Hello }\CharTok{$\{}\DataTypeTok{NAME}\CharTok{\}}
\end{Highlighting}
\end{Shaded}

输出结果为:

\begin{Shaded}
\begin{Highlighting}[]
\ExtensionTok{$}\NormalTok{ make }\AttributeTok{{-}f}\NormalTok{ anyfile hello}
\ExtensionTok{Hello}\NormalTok{ xiaoming}
\end{Highlighting}
\end{Shaded}

\texttt{make} 的开发目的是跟踪目标之间的相互依赖关系,并在任何文件发生更改时,仅执行必要的步骤。例如,如果你已经有一个基因组索引,\texttt{make} 命令将跳过索引步骤。

\begin{Shaded}
\begin{Highlighting}[]
\CommentTok{\# Set the prefix from tabs to \textgreater{}}
\DataTypeTok{.RECIPEPREFIX} \CharTok{=}\StringTok{ \textgreater{}}

\DecValTok{counts.txt:}
\NormalTok{\textgreater{} echo 100 \textgreater{} counts.txt}

\DecValTok{names.txt:}
\NormalTok{\textgreater{} echo Joe \textgreater{} names.txt}

\DecValTok{results.txt:}\DataTypeTok{ counts.txt names.txt}
\NormalTok{\textgreater{} cat counts.txt names.txt \textgreater{} results.txt}
\end{Highlighting}
\end{Shaded}

在上面的 Makefile 中,输入 \texttt{make\ results.txt} 将首先生成 \texttt{counts.txt} 和 \texttt{names.txt},如果它们不存在的话。然后,它将从这两个文件生成 \texttt{results.txt}。如果你再次运行 \texttt{make},它将不会执行任何操作,因为所有文件已经存在。

\section{makefile 的小技巧}\label{makefile-ux7684ux5c0fux6280ux5de7}

\subsection{默认设置}\label{ux9ed8ux8ba4ux8bbeux7f6e}

在 Makefile 中添加上这些设置后可以让 makefile 更加强大:

\begin{Shaded}
\begin{Highlighting}[]
\DataTypeTok{.RECIPEPREFIX} \CharTok{=}\StringTok{ \textgreater{}}
\OtherTok{.DELETE\_ON\_ERROR:}
\DataTypeTok{SHELL} \CharTok{:=}\StringTok{ bash}
\OtherTok{.ONESHELL:}
\DataTypeTok{.SHELLFLAGS} \CharTok{:=}\StringTok{ {-}eu {-}o pipefail {-}c}
\DataTypeTok{MAKEFLAGS} \CharTok{+=}\StringTok{ {-}{-}warn{-}undefined{-}variables {-}{-}no{-}print{-}directory}
\end{Highlighting}
\end{Shaded}

\texttt{.DELETE\_ON\_ERROR:}

这个特殊目标告诉 \texttt{make} 在命令执行失败时删除生成的目标文件,以防止生成不完整或损坏的文件。
\texttt{SHELL\ :=\ bash}

这行代码指定 \texttt{make} 使用 \texttt{bash} 作为默认的 shell 来执行命令。默认情况下,\texttt{make} 使用 \texttt{/bin/sh},但可以通过这种方式指定使用其他 shell。
\texttt{.ONESHELL:}

这个特殊目标指示 \texttt{make} 将单个目标的所有命令在同一个 shell 会话中执行,而不是为每个命令启动一个新的 shell。这对于需要在多个命令之间共享环境或状态的情况很有用。
\texttt{.SHELLFLAGS\ :=\ -eu\ -o\ pipefail\ -c}

这些标志用于配置 \texttt{bash} 的行为:
\texttt{-e}:如果任何命令失败(返回非零状态),则 \texttt{bash} 退出。
\texttt{-u}:使用未定义的变量时,\texttt{bash} 会报错并退出。
\texttt{-o\ pipefail}:如果管道中的任何命令失败,整个管道返回失败状态。
\texttt{-c}:从字符串中读取命令并执行。
\texttt{MAKEFLAGS\ +=\ -\/-warn-undefined-variables\ -\/-no-print-directory}

\texttt{-\/-warn-undefined-variables}:警告使用未定义的变量,以帮助捕获拼写错误或遗漏的变量定义。
\texttt{-\/-no-print-directory}:禁用 \texttt{make} 在递归调用时打印目录信息,这可以使输出更简洁。

\subsection{``试运行''模式}\label{ux8bd5ux8fd0ux884cux6a21ux5f0f}

在 bash 命令中,命令对不对,我们可以运行一下看看,如果不对,我们可以再次运行。但是在 makefile 中,我们不希望这样,我们希望一次就对,所以我们可以使用 \texttt{-n} 参数来进行``试运行''模式。这样 makefile 会输出将要执行的命令,但是不会真正执行。这样可以帮助我们检查命令是否正确。

\begin{Shaded}
\begin{Highlighting}[]
\ExtensionTok{$}\NormalTok{ make hello}
\ExtensionTok{Hello}\NormalTok{ xiaoming}

\ExtensionTok{$}\NormalTok{ make hello }\AttributeTok{{-}n}
\BuiltInTok{echo}\NormalTok{ Hello xiaoming}
\end{Highlighting}
\end{Shaded}

\subsection{变量命名}\label{ux53d8ux91cfux547dux540d}

在 Makefile 中可以通过 \texttt{?=} 这样的方式预设变量:

\begin{Shaded}
\begin{Highlighting}[]
\DataTypeTok{.RECIPEPREFIX} \CharTok{=}\StringTok{ \textgreater{}}

\DataTypeTok{NAME} \CharTok{?=}\StringTok{ xiaoming}

\DecValTok{hello:}
\NormalTok{\textgreater{} @echo Hello }\CharTok{$\{}\DataTypeTok{NAME}\CharTok{\}}
\end{Highlighting}
\end{Shaded}

这样,如果我们在命令行中没有定义 \texttt{NAME} 变量,那么 \texttt{NAME} 就会被设置为 \texttt{xiaoming}。

\begin{Shaded}
\begin{Highlighting}[]
\ExtensionTok{$}\NormalTok{ make hello}
\ExtensionTok{Hello}\NormalTok{ xiaoming}

\ExtensionTok{$}\NormalTok{ make hello NAME=lihua}
\ExtensionTok{Hello}\NormalTok{ lihua}
\end{Highlighting}
\end{Shaded}

\subsection{文本替换}\label{ux6587ux672cux66ffux6362}

在某些情况下,您想从现有字符串创建一个新字符串。

例如,你想从路径中提取目录名或文件名,如

PATH = data/reads/abc.txt
make 语法提供了函数来执行此操作,例如:

DIR = \$(dir \$\{PATH\}) 将包含 data/reads/
FNAME = \$(notdir \$\{PATH\}) 将包含 abc.txt

\begin{Shaded}
\begin{Highlighting}[]
\CommentTok{\# Sets the prefix for commands.}
\DataTypeTok{.RECIPEPREFIX} \CharTok{=}\StringTok{ \textgreater{}}

\CommentTok{\# Set a filename}
\DataTypeTok{FILE} \CharTok{=}\StringTok{ data/refs/ebola.fa.gz}

\DecValTok{demo:}

\CommentTok{\# Prints: data/refs/ebola.fa.gz}
\NormalTok{\textgreater{} @echo }\CharTok{$\{}\DataTypeTok{FILE}\CharTok{\}}

\CommentTok{\# Prints: data/refs}
\NormalTok{\textgreater{} @ echo }\CharTok{$(}\KeywordTok{dir}\StringTok{ }\CharTok{$\{}\DataTypeTok{FILE}\CharTok{\})}

\CommentTok{\# Prints: ebola.fa.gz}
\NormalTok{\textgreater{} @ echo }\CharTok{$(}\KeywordTok{notdir}\StringTok{ }\CharTok{$\{}\DataTypeTok{FILE}\CharTok{\})}

\CommentTok{\# Prints: data/refs/human.fa.gz}
\NormalTok{\textgreater{} @ echo }\CharTok{$(}\KeywordTok{subst}\StringTok{ ebola}\KeywordTok{,}\StringTok{human}\KeywordTok{,}\CharTok{$\{}\DataTypeTok{FILE}\CharTok{\})}

\CommentTok{\# Prints: data/refs/ebola.fa}
\NormalTok{\textgreater{} @echo }\CharTok{$(}\KeywordTok{patsubst}\StringTok{ \%.gz}\KeywordTok{,}\StringTok{\%}\KeywordTok{,}\CharTok{$\{}\DataTypeTok{FILE}\CharTok{\})}
\end{Highlighting}
\end{Shaded}

\begin{Shaded}
\begin{Highlighting}[]
\ExtensionTok{$}\NormalTok{ make demo}
\ExtensionTok{data/refs/ebola.fa.gz}
\ExtensionTok{data/refs/}
\ExtensionTok{ebola.fa.gz}
\ExtensionTok{data/refs/human.fa.gz}
\ExtensionTok{data/refs/ebola.fa}
\end{Highlighting}
\end{Shaded}

\section{生信流程模块化}\label{ux751fux4fe1ux6d41ux7a0bux6a21ux5757ux5316}

虽然我们都希望一个脚本能够完成所有的工作,这样是可实现的,但可能会遇到各种各样的问题,比如脚本过长,不易维护,不易复用等。因此,我们可以将脚本分解为多个模块,每个模块负责一个特定的任务。这样可以提高代码的可读性,可维护性和可复用性。

明白了需求后,按照一个良好的编写习惯,可以很好的将脚本模块化。这样可以提高代码的可读性,可维护性和可复用性。我们接下来编写一个从 SRA 数据库下载 metadata 数据的 make 脚本吧。

\subsection{step1: 默认设置}\label{step1-ux9ed8ux8ba4ux8bbeux7f6e}

\begin{Shaded}
\begin{Highlighting}[]
\CommentTok{\# Makefile customizations.}
\DataTypeTok{.RECIPEPREFIX} \CharTok{=}\StringTok{ \textgreater{}}
\OtherTok{.DELETE\_ON\_ERROR:}
\DataTypeTok{SHELL} \CharTok{:=}\StringTok{ bash}
\OtherTok{.ONESHELL:}
\DataTypeTok{.SHELLFLAGS} \CharTok{:=}\StringTok{ {-}eu {-}o pipefail {-}c}
\DataTypeTok{MAKEFLAGS} \CharTok{+=}\StringTok{ {-}{-}warn{-}undefined{-}variables {-}{-}no{-}print{-}directory}
\end{Highlighting}
\end{Shaded}

\subsection{step2: 打印帮助}\label{step2-ux6253ux5370ux5e2eux52a9}

我们总是会忘记自己的脚本是干什么的。因此,最好第一个命令是打印帮助。

\begin{Shaded}
\begin{Highlighting}[]
\CommentTok{\# Print usage information.}
\DecValTok{usage:}
\NormalTok{\textgreater{} @echo }\StringTok{"\#"}
\DecValTok{\textgreater{} @echo "\# metadata.mk:}\DataTypeTok{ downloads metadata from SRA"}
\NormalTok{\textgreater{} @echo }\StringTok{"\#"}
\NormalTok{\textgreater{} @echo }\StringTok{"\# SRA=}\CharTok{$\{}\DataTypeTok{SRA}\CharTok{\}}\StringTok{"}
\NormalTok{\textgreater{} @echo }\StringTok{"\#"}
\NormalTok{\textgreater{} @echo }\StringTok{"\# make run|keys|clean"}
\NormalTok{\textgreater{} @echo }\StringTok{"\#"}
\end{Highlighting}
\end{Shaded}

\subsection{step3: 定义变量}\label{step3-ux5b9aux4e49ux53d8ux91cf}

刚开始可以不需要这一步,等到足够熟练使用后,可以将一些常用的变量定义在这里。

\begin{Shaded}
\begin{Highlighting}[]
\CommentTok{\# Sets the default target.}
\DataTypeTok{SRA} \CharTok{?=}\StringTok{ PRJEB31790}
\end{Highlighting}
\end{Shaded}

\subsection{step4: 添加代码}\label{step4-ux6dfbux52a0ux4ee3ux7801}

这里添加一下下载,提取信息,清理的代码。

\begin{Shaded}
\begin{Highlighting}[]
\DecValTok{run:}
\NormalTok{\textgreater{} @bio search }\CharTok{$\{}\DataTypeTok{SRA}\CharTok{\}}\NormalTok{ {-}H {-}{-}csv {-}{-}all \textgreater{} }\CharTok{$\{}\DataTypeTok{SRA}\CharTok{\}}\NormalTok{.csv}

\DecValTok{keys:}\DataTypeTok{ }\CharTok{$\{}\DataTypeTok{SRA}\CharTok{\}}\DataTypeTok{.csv}
\NormalTok{\textgreater{} @cat }\CharTok{$\{}\DataTypeTok{SRA}\CharTok{\}}\NormalTok{.csv | csvcut {-}c run\_accession,sample\_title}

\DecValTok{clean:}
\NormalTok{\textgreater{} @rm {-}f }\CharTok{$\{}\DataTypeTok{SRA}\CharTok{\}}\NormalTok{.csv}
\end{Highlighting}
\end{Shaded}

\subsection{step5: 运行命令}\label{step5-ux8fd0ux884cux547dux4ee4}

把上述所有代码放到一个文件中,然后运行 make 命令。

\begin{Shaded}
\begin{Highlighting}[]
\CommentTok{\# 使用默认变量}
\FunctionTok{make} \AttributeTok{{-}f}\NormalTok{ metadata.mk run}

\CommentTok{\# 使用自定义变量}
\FunctionTok{make} \AttributeTok{{-}f}\NormalTok{ metadata.mk run SRA=PRJNA932187}

\CommentTok{\# 提取关键信息}
\FunctionTok{make} \AttributeTok{{-}f}\NormalTok{ metadata.mk keys}
\CommentTok{\# 也可以直接运行,因为定义了依赖,会默认先运行 run,然后运行 keys}

\CommentTok{\# 清理}
\FunctionTok{make} \AttributeTok{{-}f}\NormalTok{ metadata.mk clean}
\end{Highlighting}
\end{Shaded}

\section{拓展}\label{ux62d3ux5c55}

我经常使用的软件 \texttt{bio} :
\url{https://www.bioinfo.help/index.html}
这个软件有很多我们常用的生信工具。我建议你使用它!

下载软件后,运行下面这个代码会下载很多的 makefile 脚本,熟悉使用它们,你会发现生信分析变得更加简单。

\begin{Shaded}
\begin{Highlighting}[]
\ExtensionTok{bio}\NormalTok{ code}
\end{Highlighting}
\end{Shaded}


\bibliography{book.bib,packages.bib}

\backmatter
\printindex

\end{document}
